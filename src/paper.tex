\documentclass{article}
\setlength{\parindent}{0pt}
\setlength{\parskip}{0.5em}

\usepackage{biblatex}
\addbibresource{references.bib}

\usepackage{amsmath}
\usepackage{amssymb}
\usepackage{listings}

\DeclareMathOperator{\acc}{value}

\usepackage{mathtools}
\usepackage{tikz}
\usetikzlibrary{automata, positioning, arrows, fit, matrix, shapes.geometric, shapes.misc, calc}
\pgfdeclarelayer{background}
\pgfdeclarelayer{foreground}
\pgfsetlayers{background,main,foreground}

\newcommand{\Ex}{\textbf{Npr.}:\ }
\newcommand{\Special}[1]{\textbf{#1}}
\newcommand{\FIRST}{\textsc{FIRST}}
\newcommand{\FOLLOW}{\textsc{FOLLOW}}
\newcommand{\Grammar}{G}
\newcommand{\Terminals}{T}
\newcommand{\NonTerminals}{N}
\newcommand{\Productions}{P}
\newcommand{\StartSymbol}{S}
\newcommand{\Empty}{\varnothing}
\newcommand{\Null}{\varepsilon}
\newcommand{\Language}[1]{\mathcal{L}(#1)}
\newcommand{\Automaton}[1]{\mathcal{M}(#1)}
\newcommand{\Str}[1]{\text{\textquotedbl\texttt{#1}\textquotedbl}}
\newcommand{\Char}[1]{\texttt{#1}}
\newcommand{\Arrow}{\Coloneqq}
\newcommand{\Derive}{\Rightarrow}
\newcommand{\DeriveStar}{\xRightarrow{\ast}}
\newcommand{\DerivePlus}{\xRightarrow{+}}
\newcommand{\Seq}{\ }
\newcommand{\Union}{\mathrel{|}}
\newcommand{\Kleene}[1]{#1^\ast}
\newcommand{\Rep}[2]{#1^#2}
\newcommand{\Opt}[1]{#1?}
\newcommand{\KleenePlus}[1]{#1^+}


\begin{document}
\Special{Meta:} Simboli so omejeni na (pod)poglavja.

\section{Gramatike}
Gramatika je tuple $\Grammar = (\NonTerminals, \Terminals, \Productions, \StartSymbol)$, kjer je $\NonTerminals$ končna množica neteminalnih simbolov, $\Terminals$ končna množica terminalnih simbolov, $\Productions$ je končna relacija produkcijskih pravil $\NonTerminals \times \Kleene{(\NonTerminals \cup \Terminals)}$ in $\StartSymbol \in \NonTerminals$ je začetni simbol.
Tukaj $\Kleene{(\NonTerminals \cup \Terminals)}$ pomeni seznam terminalnih in neterminalnih simbolov (prazen seznam je $\Null$).

\Ex
\begin{align*}
  \Grammar_1 &= (\NonTerminals_1, \Terminals_1, \Productions_1, \StartSymbol_1) \\
  \NonTerminals_1 &= \{L\}\\
  \Terminals_1 &= \{i\}\\
  \Productions_1 &= \{ (L, (i, L)), (L, \Null) \}\\
  \StartSymbol_1 &= L
\end{align*}

Produkcijska pravila pišemo z operatorjem $(\Arrow)$, na levi strani je vedno en neterminalen simbol na desni pa je lahko seznam terminalnih in neterminalnih simbolov.

\Special{Meta:} Simbol na levi strani prvega produkcijskega pravila je začetni simbol

\Ex
\begin{equation}
  \tag{$\Grammar_1$}
  \label{g:listi}
  \begin{aligned}
    L &\Arrow i \Seq L\\
    L &\Arrow \Null
  \end{aligned}
\end{equation}

Produkcijska pravila, ki imajo na levi strani enak neterminalen simbol lahko združimo z unijo.
\Ex
\begin{align*}
  L &\Arrow i \Seq L \Union \Null\\
\end{align*}
Gramatike so sistem prepisovanja.
V seznamu terminalnih in neterminalnih simbolov, lahko neterminalne simbole zamenjamo z eno izmed desnih strani produkcijskega pravila, ki ima ta simbol na levi strani.
Postopek je podoben, kot pri vstavljanju v enačbo, le da lahko imamo več različnih možnosti (več desnih strani).

\Ex
  \begin{align*}
    y &= x + x\\
    x &= 1
  \end{align*}
  Če bi imeli poleg tega enačbo $x = 2$, bi nastal problem ($1 = 2$).

  \begin{equation*}
    y = x + x = 1 + x = 1 + 1
  \end{equation*}

  \begin{equation}
    \tag{$\Grammar_2$}
    \label{g:plusab}
    \begin{aligned}
      E &\Arrow N \Seq p \Seq N \Union \Null\\
      N &\Arrow a \Union b
    \end{aligned}
  \end{equation}

Če smo zamenjali samo en neterminalen simbol potem uporabimo operator $(\Derive)$, če smo jih zamenlali več $(\DerivePlus)$, če smo jih zamenjali več ali nobenega pa $(\DeriveStar)$.
  \begin{align*}
    E &\Derive N \Seq p \Seq N \Derive a \Seq p \Seq N \Derive a \Seq p \Seq b \\
    E &\DerivePlus a \Seq p \Seq b\\
    E &\DeriveStar E\\
    E &\DeriveStar a \Seq p \Seq b
  \end{align*}

Stavčne oblike gramatike, so vsi seznami neterminalnih ali terminalnih simbolov $\alpha \in \Kleene{(\NonTerminals \cup \Terminals)}$, kjer $\StartSymbol \DeriveStar \alpha$.
Seznam vseh stavčnih oblik dobimo tako, da gremo sistematično skozi vse možne izpeljave.

\Ex

  \begin{align*}
    &L \\
    &L \Derive i \Seq L \Derive i \Seq i \Seq L \Derive \dots \\
    &\Null \Derive i \Seq \Null \Derive i \Seq i \Seq \Null \Derive \dots \\
  \end{align*}
Za \ref{g:listi} so vse stavčne oblike $\{L, i \Seq L, i \Seq i \Seq L, \dots, \Null, i, i \Seq i, \dots\}$.

  \begin{align*}
    &E \\
    &E \Derive N \Seq p \Seq N \\
    &E \Derive N \Seq p \Seq N \Derive a \Seq p \Seq N\\
    &E \Derive N \Seq p \Seq N \Derive a \Seq p \Seq N \Derive a \Seq p \Seq a\\
    &E \Derive N \Seq p \Seq N \Derive a \Seq p \Seq N \Derive a \Seq p \Seq b\\
    &E \Derive N \Seq p \Seq N \Derive b \Seq p \Seq N\\
    &E \Derive N \Seq p \Seq N \Derive b \Seq p \Seq N \Derive b \Seq p \Seq a\\
    &E \Derive N \Seq p \Seq N \Derive b \Seq p \Seq N \Derive b \Seq p \Seq b\\
  \end{align*}
Za \ref{g:plusab} so vse stavčne oblike $\{E, N \Seq p \Seq N, a \Seq p \Seq N, b \Seq p \Seq N, N \Seq p \Seq a, N \Seq p \Seq b, a \Seq p \Seq a, a \Seq p \Seq b, b \Seq p \Seq a, b \Seq p \Seq b \}$.

Stavki gramatike, so vsi seznami terminalnih simbolov $w \in \Kleene{\Terminals}$, kjer $\StartSymbol \DeriveStar w$.
Seznam vseh stavkov dobimo tako, da gremo sistematično skozi vse možne izpeljave in izberemo samo tiste stavčne oblike, ki vsebujejo samo terminalne simbole.

\Ex

Za \ref{g:listi} so vsi stavki $\{\Null, i, i \Seq i, \dots\}$.

Za \ref{g:plusab} vsi stavki $\{a \Seq p \Seq a, a \Seq p \Seq b, b \Seq p \Seq a, b \Seq p \Seq b \}$.

Jezik, ki ga gramatika opisuje je množica vseh stavkov:
\begin{equation*}
  \Language{\Grammar} = \{ w \in \Kleene{T} \mid \StartSymbol \DeriveStar w\}
\end{equation*}

\Ex
  \begin{align*}
    \Language{\Grammar_1} &= \{\Null, i, i \Seq i, \dots\}\\
    \Language{\Grammar_2} &= \{a \Seq p \Seq a, a \Seq p \Seq b, b \Seq p \Seq a, b \Seq p \Seq b \}
  \end{align*}

\subsection{Razpoznavanje}

Preveriti želimo ali je nek seznam terminalov $u \in \Kleene{T}$ v jeziku za gramatiko $\Grammar$, torej $u \in \Language{\Grammar}$.

\Ex
  \begin{equation*}
    i \Seq i \in \Language{\Grammar_1}
  \end{equation*}

  Ker $i \Seq i \in \{\Null, i, i \Seq i, \dots\}$ je odgovor v tem primeru pritrdilen.

Problem je, da moramo sistematično iti skozi vse stavke.

Rešitev je, da poskušamo iz $\StartSymbol$ izpeljati $u$, če je to mogoče potem $\StartSymbol \DeriveStar u$ in $u \in \Language{\Grammar}$ (po definiciji).
Ker $u$ poznamo v naprej stavčnih oblik iz katerih $u$ ne gre izpeljati ne rabimo upoštevati.
Za določene jezike velja, da je na vsaki točki izpeljave možna samo ena stavčna oblika.

\Ex
  \begin{equation*}
    L \Derive i \Seq L \Derive i \Seq i \Seq L \Derive i \Seq i \Seq \Null
  \end{equation*}
  Torej je odgovor pritrdilen.

Problem nastane, ko moramo izbrati med večimi desnimi stranmi.

\Ex
  \begin{equation*}
    b \Seq p \Seq a \in \Language{\Grammar_2}
  \end{equation*}

  \begin{align*}
    &E \\
    &E \Derive N \Seq p \Seq N \\
    &E \Derive N \Seq p \Seq N \Derive a \Seq p \Seq N\\
    &E \Derive N \Seq p \Seq N \Derive b \Seq p \Seq N
  \end{align*}

Tukaj imamo dve možnosti ali uporabimo $N \Arrow a$ ali pa $N \Arrow b$.
Rešitev je, da $a \Seq p \Seq N$ in $b \Seq p \Seq N$ primerjamo s $b \Seq p \Seq a$.
Očitno je, da je pravilna izbira $N \Arrow b$.

V nadalevanju se bomo omelili na takšne gramatike, kjer je mogoče izbrati pravilno desno stran s primerjavo enega samega terminalnega simbola.
Prav tako bomo omejili vrstni red prepisovanja, vedno bomo najprej zamenjali najbolj levi neterminalni simbol.

\subsection{\FIRST}


\subsection{\FOLLOW}


\printbibliography
\end{document}
